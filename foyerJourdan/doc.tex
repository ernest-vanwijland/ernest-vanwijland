\documentclass[french,12pt,amstex,a4paper]{article}
\usepackage{xcolor}
\usepackage{listings}
\usepackage{graphicx}
\usepackage{amsfonts}
\usepackage{amsmath}
\usepackage[utf8]{inputenc}
\usepackage[T1]{fontenc}
\usepackage[francais]{babel}
\usepackage{float}
\usepackage{hyperref}

\usepackage{tgpagella}

\begin{document}

\begin{center}
{\bf Confinés hors de notre foyer}\\
%{\it Ernest van Wijland, le 29 octobre 2020}\\
\end{center}

Qu'elles étaient agréables, ces douces soirées de fin d'été passées à boire du thé sur les tables de Jourdan, l'herbe verdoyante sous nos semelles, les étoiles scintillantes au-dessus de nos têtes. Néanmoins, nous nous trouvâmes bien dépourvus, quand la bise fut venue.\\

Certes, nous disposions toujours des spacieuses et confortables cuisines, mais comment faire face aux vociférations de ceux qui désirent dormir et sont génés par le bruit ? Ils ont tout à fait raison. Ainsi, à la recherche d'un nouveau foyer pour héberger notre vie nocturne, nous crûmes dans un premier temps trouver l'endroit idoine ; une pièce, munie de canapés, de chaises, de tables et d'un babyfoot, et isolée des chambres peuplées des victimes de Morphée. C'est le foyer de Jourdan.\\

Nous découvrîmes rapidement qu'il fermait cependant à une heure du matin, pour rouvrir à sept heures, sans pourtant qu'il soit nettoyé, cette fermeture étant purement d'ordre réglementatif. On pouvait s'en accomoder, c'était déjà ça, et puis il ne faisait pas si froid dehors, on pouvait terminer la soirée sur l'herbe, à condition de porter un manteau, ou bien boire une tisane en silence dans les cuisines.\\

À l'heure d'un confinement total, où nous ne pouvons plus sortir que pour faire des courses à moins d'un kilomètre de la résidence, et où nous sommes éloignés de notre foyer, celui de notre famille, parfois pour plusieurs semaines, est-il réellement nécessaire et raisonnable de maintenir cette fermeture automatique du foyer ? Que nous reste-t-il si chaque soir nous ne pouvons nous réunir, même à quelques-uns, que dans le silence ou le froid ? Qu'à peine l'heure du dîner passée, c'est pratiquement un couvre-feu forcé au sein de la résidence qui nous est imposé ?

Ainsi, si la contrainte sur les heures d'ouvertures du foyer pouvait être allégée, cela constituerait une réelle amélioration de nos conditions de confinement. Nous vous en serions extrêmement reconnaissant si vous pouviez envisager cette possibilité.

\end{document}










































